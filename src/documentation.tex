\documentclass[a4paper,12pt]{article}
\usepackage[utf8]{inputenc}
\usepackage[russian]{babel}
\usepackage{geometry}
\usepackage{hyperref}
\usepackage{listings}
\usepackage{xcolor}

\geometry{left=2cm,right=2cm,top=2cm,bottom=2cm}

\lstset{
    language=C++,
    basicstyle=\ttfamily\small,
    keywordstyle=\color{blue},
    commentstyle=\color{green!50!black},
    stringstyle=\color{red},
    numbers=left,
    numberstyle=\tiny,
    stepnumber=1,
    numbersep=5pt,
    breaklines=true,
    frame=single,
    tabsize=2
}

\title{Документация проекта 3DViewer v2.0}
\author{Разработчики: sionapae, fearowpi}
\date{\today}

\begin{document}

\maketitle
\tableofcontents
\newpage

\section{Общее описание проекта}

\textbf{3DViewer v2.0} — это приложение для визуализации трёхмерных моделей в формате \texttt{.obj}.  
Программа отображает модель в каркасном виде (wireframe), позволяет выполнять аффинные преобразования (перемещение, вращение, масштабирование), выбирать тип проекции (центральная или параллельная) и изменять параметры визуализации (цвета, тип линий, тип вершин).  

Проект реализован на \textbf{C++20} с использованием библиотеки \textbf{Qt 6} и принципов \textbf{ООП}.  
Основное внимание уделено архитектуре, основанной на паттернах проектирования \textbf{Facade}, \textbf{Strategy} и \textbf{Singleton}.

\section{Структура проекта}

\subsection{Директории и файлы}

\begin{itemize}
    \item \texttt{gui/3d/} — реализация графического интерфейса и визуализации:
    \begin{itemize}
        \item \texttt{mainwindow.h / mainwindow.cpp} — главное окно и логика интерфейса пользователя.
        \item \texttt{glwidget.h / glwidget.cpp} — OpenGL-виджет для рендеринга модели.
        \item \texttt{projection\_strategy.h} — стратегии построения проекций.
        \item \texttt{viewer\_facade.h} — фасад для связи GUI и рендерера.
        \item \texttt{settings\_manager.h / settings\_manager.cpp} — управление пользовательскими настройками (Singleton).
    \end{itemize}

    \item \texttt{model/} — модуль бизнес-логики:
    \begin{itemize}
        \item \texttt{model.h / model.cpp} — загрузка и хранение данных 3D-модели (.obj).
    \end{itemize}

    \item \texttt{tests/} — модуль модульного тестирования (Google Test):
    \begin{itemize}
        \item \texttt{test\_parser.cpp} — тесты парсера OBJ-файлов и функций аффинных преобразований.
    \end{itemize}

    \item \texttt{Makefile} — скрипт сборки проекта.
\end{itemize}

\section{Архитектура приложения}

Приложение построено по принципу разделения ответственности между модулями:

\begin{itemize}
    \item \textbf{MainWindow} отвечает за пользовательский интерфейс и обработку сигналов.
    \item \textbf{Viewer\_facade} инкапсулирует доступ к логике рендеринга и модели, скрывая детали реализации.
    \item \textbf{GLWidget} реализует визуализацию модели с помощью OpenGL.
    \item \textbf{Model} отвечает за парсинг и хранение данных модели (.obj).
    \item \textbf{Settings\_manager} обеспечивает централизованное хранение и загрузку пользовательских настроек.
\end{itemize}

\subsection{Класс MainWindow}

\begin{itemize}
    \item Наследует \texttt{QMainWindow}.
    \item Обеспечивает взаимодействие пользователя с 3D-сценой.
    \item Основные функции:
    \begin{itemize}
        \item Открытие и загрузка моделей;
        \item Синхронизация ползунков и полей ввода для управления трансформациями;
        \item Изменение цветов, типов и размеров линий и вершин;
        \item Переключение типа проекции.
    \end{itemize}
\end{itemize}

\subsection{Класс GLWidget}

\begin{itemize}
    \item Наследует \texttt{QOpenGLWidget}.
    \item Реализует рендеринг 3D-модели и вычисление аффинных матриц:
    \begin{itemize}
        \item \texttt{rotationMatrix()} — матрица вращения по осям X, Y, Z;
        \item \texttt{scalingMatrix()} — матрица масштабирования;
        \item \texttt{translationMatrix()} — матрица перемещения.
    \end{itemize}
    \item Обрабатывает события мыши (\texttt{mouseMoveEvent()}, \texttt{wheelEvent()}).
\end{itemize}

\subsection{Класс Model}

\begin{itemize}
    \item Отвечает за парсинг файлов \texttt{.obj}.
    \item Хранит массивы вершин и индексов.
    \item Реализует нормализацию координат модели и обработку некорректных данных.
\end{itemize}

\subsection{Класс Settings\_manager}

\begin{itemize}
    \item Реализует паттерн Singleton.
    \item Сохраняет пользовательские настройки (цвет фона, тип линии, тип проекции и т.д.).
    \item Использует Qt-класс \texttt{QSettings} для сохранения значений между сессиями.
\end{itemize}

\subsection{Класс Viewer\_facade}

\begin{itemize}
    \item Служит фасадом между \texttt{MainWindow} и \texttt{GLWidget}.
    \item Скрывает сложность управления сценой, предоставляя простой интерфейс:
    \begin{itemize}
        \item \texttt{loadModel()};
        \item \texttt{set\_background\_color()};
        \item \texttt{update\_View()}.
    \end{itemize}
\end{itemize}

\section{Паттерны проектирования}

В проекте 3DViewer v2.0 применено несколько ключевых паттернов проектирования.

\subsection{Фасад (Facade)}

\textbf{Тип:} структурный паттерн.

Класс \texttt{Viewer\_facade} инкапсулирует взаимодействие между графическим интерфейсом (\texttt{MainWindow}) и рендерером (\texttt{GLWidget}).  
Это позволяет GUI не зависеть от деталей OpenGL и структуры модели.



\subsection{Стратегия (Strategy)}

\textbf{Тип:} поведенческий паттерн.

Реализован в \texttt{projection\_strategy.h}.  
Определяет интерфейс \texttt{Projection\_strategy} с методом \texttt{create(aspect)}, который реализуют два класса:
\begin{itemize}
    \item \texttt{Parallel\_strategy} — ортографическая проекция;
    \item \texttt{Centrall\_strategy} — перспективная проекция.
\end{itemize}



\subsection{Одиночка (Singleton)}

\textbf{Тип:} порождающий паттерн.

Класс \texttt{Settings\_manager} гарантирует, что существует единственный объект, управляющий настройками программы.  
Это обеспечивает единый доступ к конфигурации интерфейса и параметрам рендеринга.





\section{Сборка и запуск проекта}

\textbf{Сборка программы:}
\begin{lstlisting}
make all
\end{lstlisting}

\textbf{Запуск:}
\begin{lstlisting}
./viewer
\end{lstlisting}

\textbf{Очистка:}
\begin{lstlisting}
make clean
\end{lstlisting}

\section{Заключение}

Приложение \textbf{3DViewer v2.0} реализует визуализацию 3D-моделей в формате \texttt{.obj}, поддержку основных аффинных преобразований и гибкую настройку отображения.  
Использование паттернов \textbf{Facade}, \textbf{Strategy}, \textbf{Singleton} и архитектуры \textbf{MVC} обеспечило модульность, простоту сопровождения и расширяемость кода.  
Проект демонстрирует применение современных подходов к проектированию программ на языке C++ с использованием Qt и OpenGL.

\end{document}
